\chapter{Muon Simulation} \label{sec:simulation}

Each analysis lacking in calibration measurements to perform a data-driven approach to analyze the experimental data requires simulations.
These simulations can be deterministic after a given initial condition, e.g. by solving the differential equations describing the processes iteratively, which is used in fluid dynamics.
However, high-energy particles behave stochastically during their propagation, choosing from different interactions and producing further stochastic particles.
A deterministic approach for particle simulations would mean to average out the stochastic nature of the particle behaviors.
This can be used to produce distributions of the incoming particle flux.
But also at the edges of the kinematic regions or in the far tail of a distribution, this approach is limited to bin widths introduced when discretizing the differential equations.
Besides, experiments in particle or astroparticle physics need to simulate individual event signatures they measure in their detector and therefore Monte-Carlo techniques are used.
Thereby, each propagation step is randomly sampled according to the physical distribution.

The name Monte-Carlo is named after the capital of Monaco which is known for its casinos and gambling, where the decision of winning or losing is random and determined e.g. by rolling dice.
Transfered to particle physics, the full event simulation consists of the combination of these single random decisions for each propagation step.
This allows to fill out the high dimensional phase space of possible event signatures by simulating a huge amount of Monte-Carlo events, which can result in billions of simulated events.

The benefit of Monte-Carlo simulations representing the whole phase space of possible event signatures can be explained by Monte-Carlo integration.
For high dimensional integration problems, deterministic approaches, using for example grid-based approaches, get outperformed by Monte-Carlo techniques, which are independent of the dimensions and the accuracy scale only with the square root of the sampling points.
While introducing an additional parameter into a high dimensional Monte-Carlo simulation, the number of simulated events required for an accurate description of the allowed phase space, does not increase.
Even the edge cases of the phase space are therefore included, which is important in particular as the amount of the interesting signal events in particle physic experiments are typically many orders of magnitude below the number of background events.

In many particle and astroparticle physics experiments, atmospheric muons are the dominant type of background although the detectors are most often located underground to reduce already a huge amount of these disturbing events.
For an accurate description of muon events that can, due to their stochastic behavior, mime signal-like events, even the rarest interactions need to be simulated with a decent amount of statistics.
A large amount of Monte-Carlo simulations is therefore required to achieve this accuracy and a trade-off between the performance of the simulation and its precision has to be made, adapted to each detector resolution.
Also, tools have to be developed to match both requirements with the specialty for muons to propagate through large volumes.

%

\section{Muon Simulation Tools}

There are multiple simulation tools that can propagate muons.
Most of them are based on the methods for the interactions and the average energy loss developed in \textit{MUDEDX} \cite{Lohmann85} and for the energy thresholds dividing continuous and stochastic energy losses, introduced in \textit{PROPMU} \cite{Lipari91}.
A similar version of the latter is used in KM3Net, called \textit{PropaMuon} \cite{Km3Net2020gSeaGen}.
Due to the deviations observed between these two frameworks \cite{Desiati01ICRC}, mainly driven by the difference in their algorithmic approach, further advanced tools were developed.
Two examples of these frameworks, that are still maintained and further developed is \textit{MUM} \cite{Sokalski01MUM, Bugaev00MUM}, used in the Baikal experiment \cite{Baikal97}, and the most widely used muon propagator \textit{MUSIC} \cite{Antonioli97, Kudryavtseva99, Kudryavtsev09}, which is used in most underground experiments like Kamiokande \cite{HyperKamiokande18}.

These simulation tools are mainly used to propagate atmospheric muons from the surface to the underground detector.
The muon generation in the air shower and their propagation in the atmosphere to the surface are performed by air shower frameworks like CORSIKA \cite{CORSIKA, Engel19}, AIRES \cite{Sciutto19AIRES}, or ZHS \cite{Zas92ZHS} for homogeneous media.
To generate a muon flux near the detector region and skip the runtime consuming part of the air shower simulation, tools like MUPAGE \cite{Carminati09} and MuonGun (based on \cite{Becherini06}) used in the neutrino telescopes KM3NeT and IceCube respectively were developed, parameterizing the energy and zenith dependent flux of the muon bundles in certain depths.

An alternative approach to estimate the atmospheric muon flux near the detector is by using the Matrix cascade equation and solve the differential equation of the shower development, like in CONEX \cite{Bergmann07CONEX}, MCEq \cite{Fedynitch15MCEq}, or EMCa \cite{MeighenBerger19EMCa, MeighenBerger19ICRC}, where the latter is specialized in the calculation of electromagnetic component.
Out of the estimated flux distribution, individual muons can be sampled to be propagate inside the detector.
However, since the differential equations have to get discretized, the resolution of the simulated muon flux depends on the binning, and muons in the far tails of the flux distribution can not be described accurately.

An alternative approach to overcome the runtime intensive production of large air shower simulation datasets to describe the event signature, but still using Monte-Carlo methods is the backward Monte-Carlo approach by PUMA \cite{Niess17}.
Instead of forward propagating the muons from the air shower to the detector, they get propagated backward from the detector to their initial state in the air shower.
This is mainly used when the forward propagation becomes extremely inefficient, e.g. if the detection area is small, while the muons can originate from a wide field of the sky, like in the muon tomography of pyramids.

The simulation of particles inside the detector is nearly always performed by the simulation framework GEANT4 \cite{Agostinelli03, Allison06, Allison16, GEANT4}.
There, the most accurate description of physical models and materials is given.
However, the design of the simulation architecture and algorithm is optimized for detectors of the size of $\mathcal{O}(\SI{10}{m})$ and already the propagation over distances of \SI{100}{m} becomes extremely inefficient.
For larger detectors, like neutrino telescopes, dedicated tools to propagate particles through large volumes, as described before, are required.

In this work, an alternative to the previously mentioned muon simulation tools, the simulation library \textit{PROPOSAL}, propagating leptons and high energy photons, has been further developed.

%

\section{The Leptonpropagator PROPOSAL}

PROPOSAL stands for \textit{\textbf{PR}opagator with \textbf{O}ptimal \textbf{P}recision and \textbf{O}ptimized \textbf{S}peed for \textbf{A}ll \textbf{L}eptons} and is a C\texttt{++} and python library for the Monte-Carlo simulation of high energy particles \cite{Koehne13PROPOSAL}.
Focusing on the electroweak interactions of leptons and high energy photons, PROPOSAL is a simulation library for Monte-Carlo propagation providing multiple selectable interaction and scattering models.
It was initially designed to propagate muons through large volumes of media and calculate the tau decay for the IceCube Neutrino Observatory.
After a couple of restructuring cycles and further enhancements and improvements PROPOSAL is now a modular simulation library used in several simulation frameworks of astroparticle physics experiments.
For the final calculations and plots of this work, version 7.1.0 was used.

PROPOSAL is a free open-source software with an LGPL License and is being developed on GitHub\footnote{see \url{https://github.com/tudo-astroparticlephysics/PROPOSAL/}}.
It can be installed as a classic \texttt{cmake} project, using the package manager conan\footnote{\url{https://conan.io/center/proposal}} for the C\texttt{++} library, or pip\footnote{\url{https://pypi.org/project/proposal/}} for the python library.
It provides the full track of a propagated particle, single propagation steps or just the theoretical description provided by the cross-sections for fine-tunable settings.
These settings are mainly defined by the input particle, the cross-sections and their parametrizations, the medium, and the cuts of the energy loss from which the integrals for the individual propagation steps can be calculated, which is described in the following sections.

%

\subsection{Historical Overview}

PROPOSAL is the successor of MMC (Muon Monte-Carlo) \cite{Chirkin03PhD, Chirkin04MMC}, written in Java, which was designed to propagate muons efficiently through large volumes of ice for the AMANDA and IceCube detector.
There were two different requirements for the muon simulations of neutrino telescopes.
First, a highly performant muon propagation through tens of kilometers of ice to the detector to obtain a sufficiently accurate muon spectrum at the detector.
Second, a precise muon simulation inside the detector region providing the energy losses for further steps of the simulation chain.
Due to the limited detector resolution, these experiments just differentiate between an electromagnetic or hadronic cascade going out of the interaction and are not interested in an accurate sampling of the secondary particles.
The propagation of taus was also included while not being the focus in the development.
At the energies relevant for IceCube, taus almost immediately decay, which is trivial to calculate when approximating any hadronic decay channel as two-body decay.
The electron propagation was not included, since this is equivalent to an electromagnetic cascade.

Next to the efficient propagation of muons, the possibility to perform simulation studies analyzing the effects of systematic uncertainties of the cross-sections was also a key target of this simulation tool.
Therefore multiple parametrizations for the bremsstrahlung and photonuclear interactions were implemented.
For pair production, there was a lack of comprehensible alternatives and for ionization, which is only dominant for lower energies where these losses can also be measured, the cross-section was already accurate enough.

During the transition of the IceCube software from Java to C\texttt{++}, also the muon and tau simulation chain was rewritten in C\texttt{++} and renamed to PROPOSAL.
This complete revision of the simulation code was done in numerous works at TU Dortmund University.
In \cite{Frantzen11Diplom, Schmitz11Diplom} the focus was on the accurate interpolation of the cross-sections while also studying the deviations between the multiple parametrizations for the interactions.
Possible calculations on GPUs were tested in \cite{Fuchs12Master} and found to be only relevant for the interpolation methods.
In \cite{Geiselbrinck13Bachelor} multiple parametrizations for the Moli\`{e}re Scattering were implemented.
All of these works were developed under the supervision of \cite{Koehne13PhD} and resulted in a publication \cite{Koehne13PROPOSAL} summarizing these works and describing the new simulation tool.

For the next years, PROPOSAL was maintained by \cite{Fuchs16PhD} and in close collaboration with the IceCube collaboration.
Meanwhile, theoretical calculations on new parametrizations of the bremsstrahlung and pair production cross-sections were developed in \cite{Menne14Master, Sandrock14Master, Soedingrekso16Master}.
This was further developed in \cite{Sandrock18PhD} leading to the publication \cite{Sandrock18nlo} of higher-order corrections of the bremsstrahlung cross-section in collaboration with the MEPhI who already developed the commonly used cross-section parametrizations.
These new cross-sections for bremsstrahlung and pair production were implemented in PROPOSAL and presented in \cite{Soedingrekso19ICRC}.

In this work, further enhancements and improvements were implemented within multiple restructuring cycles.
The first cycle of restructuring together with \cite{Dunsch18Master} introduced a more modular and polymorph structure and established PROPOSAL as a software library.
Unit tests were introduced to test each part individually and verify the reproducibility of the simulation.
Initially intended for easier testing, the so-called propagation utility was introduced collecting the integrals required during the propagation (described in \secref{sec:prop_util}) and separate them from the main propagation routine.
Also, pybindings were introduced so PROPOSAL can be used as a C\texttt{++} or python library, two of the most common programming languages in the scientific community by now.
Furthermore, the decay process was completely revised introducing a new phase space calculation for many-body decays.
The whole improvements are described in the publication \cite{Dunsch19PROPOSAL}.
With this new structure, new experiments became interests in using this simulation software, e.g. for radio neutrino astronomy \cite{Glaser20nuRadioMC}.
A first attempt to introduce neutrino propagation to be able to simulate tau regeneration was done in \cite{Franz20Bachelor}.

The interest in using PROPOSAL for the air shower simulation framework CORSIKA (c.f. section \ref{sec:corsika}) to calculate the electromagnetic shower component introduced the second round of restructuring to meet these requirements.
New cross-sections for electrons, positrons, and high energy photons were implemented in \cite{Alameddine20Master} and presented in \cite{Alameddine20PROPOSAL}.
CORSIKA introduced a new kind of use case of PROPOSAL as it wants to get the physics, e.g. the cross-sections or mean free path so that PROPOSAL is proposing an interaction step while CORSIKA is propagating the particles.
For this purpose, the propagation utilities became important and the new level of modularity was introduced.
This is mainly described in \cite{Sackel21Master}, implementing also new interpolation techniques.

Regarding the most recent developments, in \cite{Gutjahr21Master} the deflections at stochastic processes were introduced and in \cite{Bollmann21Bachelor} the produced electromagnetic air showers in the new CORSIKA framework using PROPOSAL are compared to previous versions of CORSIKA.
Jean-Marco Alameddine is the next main developer after this work and was already doing key developments in the second round of restructuring.
In the following sections, the current status of PROPOSAL is described.

%

\subsection{Medium and Component} \label{sec:medium}

In PROPOSAL, the targets, where the particles interact, are either the medium or single components, i.e. the atoms, of the medium.
These components are defined by 
\begin{itemize}
    \item the name of the component.
    \item the charge $Z$ of the component
    \item the average mass $A$ of an atom
\end{itemize}
Next to the components defined in the periodic table, there are also effective elements, like the \textit{StandardRock} with $Z=11, A=22$, which is an effective description of CaCO$_3$ and a widely used material describing rock.
Out of these adjustable parameters, the following parameters are calculated during initialization:
\begin{itemize}
    \item the average mass of a nucleon $\bar{m}_N$
    \item the elastic and inelastic constant of the radiation logarithm $B_{\text{(in)el}}$
    \item the Woods-Saxon potential (up to now, only used in the \textit{ButkevichMikheyev} parametrization)
\end{itemize}
There are several values given for the constant of the radiation logarithm which is required to describe the elastic atomic form factor (see section \ref{sec:brems_screen}).
In the Thomas-Fermi model for the electron distribution, $B_{\text{el}} \approx \num{183}$, independent of the nuclear charge but only applicable for elements with high $Z$ \cite{Bethe34a, Bethe34b}.
However, $B_{\text{el}}$ in principle depends on the nuclear charge.
A more accurate description using the Hartree-Fock model \cite{Kelner99RadLog} results in the values listed in \tabref{tab:rad_log}.
\begin{table}
    \caption{The $Z$ dependence of the elastic radiation logarithm constants in the Hartree-Fock model, calculated in \cite{Kelner99RadLog}. This parameter was originally introduced as a constant in the logarithm of the screening function $\Phi_1$ in the complete screening approximation.}
    \label{tab:rad_log}
    \begin{center}
    \begin{tabular}{cc|cc|cc|cc|cc}
        \toprule
        $Z$ & $B_{\text{el}}$ & $Z$ & $B_{\text{el}}$ & $Z$ & $B_{\text{el}}$ & $Z$ & $B_{\text{el}}$ & $Z$ & $B_{\text{el}}$ \\
        \midrule
        1 & 202.4 & 8 & 173.4 & 15 & 172.2 & 22 & 176.8 & 53 & 178.6 \\
        2 & 151.9 & 9 & 170.0 & 16 & 173.4 & 26 & 175.8 & 74 & 177.6 \\
        3 & 159.9 & 10 & 165.8 & 17 & 174.3 & 29 & 173.1 & 82 & 178.0 \\
        4 & 172.3 & 11 & 165.8 & 18 & 174.8 & 32 & 173.0 & 92 & 179.8 \\
        5 & 177.9 & 12 & 167.1 & 19 & 175.1 & 35 & 173.5 & & \\
        6 & 178.3 & 13 & 169.1 & 20 & 175.6 & 42 & 175.9 & others & 182.7 \\
        7 & 176.6 & 14 & 170.8 & 21 & 176.2 & 50 & 177.4 & & \\
        \bottomrule
    \end{tabular}
    \end{center}
\end{table}
For the constant of the inelastic radiation logarithm, PROPOSAL only differentiates between Hydrogen with $B_{\text{inel}} = 1429$ and all other elements with $B_{\text{inel}} = 1194$ \cite{Tsai74}.

The Wood-Saxon potential is defined by \cite{Butkevich02}
\begin{align}
    N_S(A) = 4 \pi \rho_0 \int\limits_{r_0}^{\infty} \frac{r^2}{1+\exp((r-r_0)/a)} \dif r
    \quad \text{with} \quad
    r_0 = 1.12 A^{\sfrac13} - 0.86 A^{-\sfrac13} .
\end{align}
% which can be written to
% \begin{align}
%     a r_0^2 \int_0^{\infty} \frac{\dif x}{1+e^x}
%         + 2a^2r_0 \int_0^{\infty} \frac{x \dif x}{1+e^x}
%         + a^3 \int_0^{\infty} \frac{x^2 \dif x}{1+e^x}
% \end{align}
For a constant $a = \SI{0.54}{fm}$ and $\rho_0 = \SI{0.17}{fm^{-3}}$, it can analytically be integrated to
\begin{align}
    N_S(A) = 4 \pi \rho_0 [a r_0^2\log(2) + a^2 r_0 \pi^2 / 6 + \sfrac32 a^3 \zeta(3)] ,
\end{align}
where $\zeta(x)$ is the Riemann zeta function.

There are also processes, where the interaction is a continuous process and does not occur at a single atom, like the density correction for ionization or the LPM and dielectric effect.
The medium constants for the ionization and density correction were already given in section \ref{sec:ioniz}.
Besides, the medium is defined by its density, which can be varied according to a given density distribution, described in section \ref{sec:geometry_density}.

From the list of components and the density, the following parameters are calculated during the initialization:
\begin{itemize}
    \item the number of protons, i.e. the charge $Z$
    \item the number of nucleons
    \item the mol density
    \item the radiation length, which is required for the LPM effect and the description of multiple scattering
\end{itemize}
The radiation length describes the distance when the electron has on average lost $1/e$ of its energy.
Since bremsstrahlung is the dominating process, the total cross-section of the electron bremsstrahlung in the complete screening case is used to calculate the radiation length
\begin{align}
    X_0 = \frac{\sum A}{\sum \sigma_{e, \text{brems.}} A} ,
\end{align}
where the sum is over each component.
% For the bremsstrahlung cross-section, the complete screening case for electrons is used, as this is accurate enough and it can be integrated analytically.

The densities for a medium are mainly taken from \cite{PDG20}.
% It is also possible to define an inhomogeneous medium where the density changes along an axis, described in section \ref{sec:geometry_density}.
However, there are processes, like the LPM effect, depending on the density of the medium in a more complex way than just a linear factor in the cross-section.
In principle, the change of the density can therefore not be treated independently of the cross-section.
Since the cross-sections are interpolated before the propagation to increase the performance, approximations are required, like assuming a locally homogeneous medium.
This issue is also faced by other simulation frameworks and finding a decent approximation or alternative treatment is still an ongoing topic of research.

%

\subsection{Geometries and Density Profiles} \label{sec:geometry_density}

A particle can be propagated through different sectors of media, each defined by a geometry and a density distribution.

There are three geometries available in PROPOSAL.
\begin{itemize}
    \item A \texttt{sphere} is defined by the coordinate of its origin and its \texttt{radius}. The sphere can also be a spherical shell by setting an \texttt{inner\_radius}, which is e.g. used to describe the different layers of the earth.
    \item A \texttt{cylinder} is defined by the same parameters of a \texttt{radius} and an \texttt{inner\_radius} as the spherical shell and the \texttt{height}. In contrast to the radius, there is no inner height of the cylindric shell.
    \item A \texttt{box} is defined by the center coordinate and the \texttt{length}, \texttt{width}, and \texttt{height}. Compared to the sphere or the cylinder, no inner lengths can be defined.
\end{itemize}
With these geometries, the environment of the simulation can be created.
Each geometry also consists of a \texttt{hierarchy} parameter, that in the case of overlapping geometries, the one with a higher hierarchy is chosen.
If both hierarchies are equal, the geometry with a higher density is used.

The density inside a geometry can also be varied along an axis using the following density profiles
\begin{itemize}
    \item A homogeneous density
    \item An exponential decreasing or increasing density
    \item A polynomial density distribution
    \item A distribution defined by interpolating splines.
\end{itemize}
Not for all interactions, the density can be treated separately.
The LPM effect has larger effects if the medium is denser and the distance to the next atom is smaller resulting in stronger influences of their wave functions with the interaction.
However, the cross-section integrals, described in \secref{sec:ecuts}, as well as the propagation integrals, described in \secref{sec:prop_util}, are interpolated before the propagation.
Therefore density effects in inhomogeneous media can only be taken into account for a reference point assuming no major changes of the density inside the geometry.
Since the density effects are mainly minor corrections or have significant effects only at higher energies, this is still applicable for most experiments.
Nevertheless, this is an important issue, which needs to get solved in the future, especially for the electron, positron, and photon propagation in the atmosphere.

%

\subsection{Particles and Secondaries} \label{sec:particle}

As the name of PROPOSAL suggests, all leptons, meaning charged leptons of all three flavors (electrons, muons, and taus) including their anti-particles and their corresponding neutrinos can be propagated.
In addition, high-energy photons can also be propagated due to similar propagation behaviors.
Also, exotic particles beyond the Standard Model of particle physics like magnetic monopoles and supersymmetric staus are implemented and can be propagated using the same propagation techniques.
Since all cross-sections are implemented adapting to the mass and charge of the primary particle, as described in chapter \ref{sec:interactions}, the same cross-sections as for the muon propagation can be used for custom particles, e.g. a muon with a mass of \SI{500}{GeV} or a milicharged particle with a charge of \num{e-3} \cite{Plestid20MiliCharged, Arguelles21MiliCharged}.
% To illustrate the change of the relevant interaction types, the average energy loss of various particles is shown in \secref{sec:dedx_various}.
The simple adaption of the particle definition is limited to the cross-sections for charged and massive particles; cross-sections for neutrinos or high energy photons can not be used for this.

In principle, every kind of particle can be propagated by changing the particle properties and using a set of suitable cross-sections given to the propagation.
For the above-mentioned particles, the relevant cross-sections or decay channels are implemented.
Further cross-sections can simply be added.
The particle properties called \texttt{ParticleDef} consist of the following parameters.
\begin{itemize}
    \item The \texttt{name} of the particle.
    \item The \texttt{mass} of the particle.
    \item The parameter \texttt{low} defining the lower limit of the particle energy to calculate the propagation integrals for the interpolation tables, thereby defining the lower limit until the particle can be propagated.
    This is usually set to the mass of the particle except for the massless particles, where it is a small value, which is necessary due to the logarithmic energy scale of the interpolation.
    For photons, the value is two times the electron mass.
    \item The \texttt{lifetime} of the particle. Stable or exotic particles have an infinite lifetime.
    \item The \texttt{charge} of the particle.
    \item The \texttt{hard\_component\_table} containing the parameters for the optional nuclear inelastic interaction. These tables are only provided for muons and taus (c.f. \secref{sec:photo_vmd}).
    \item The \texttt{decay\_table} listing the possible decay channels for the particle. This is only provided for muons and taus.
    \item The \texttt{particle\_type} is the PDG code the particle.
    \item The \texttt{weak\_partner} is the PDG code of the weak partner particle produced in a charged current weak interaction. This is only defined for leptons.
\end{itemize}
To propagate these particles, the dynamic properties are collected in the so-called \texttt{ParticleState} storing the following parameters.
\begin{itemize}
    \item The \texttt{position} of the particle.
    \item The \texttt{direction} of the particle.
    \item The \texttt{energy} of the particle. This directly also sets the \texttt{momentum} for massive particles.
    \item The simulation \texttt{time} of the particle.
    \item The \texttt{propagated\_distance} storing the distance the particle has been propagated.
    \item The \texttt{type} of the particle, i.e. an identifier using the PDG codes.
\end{itemize}
Next to the particles that can be propagated, the secondary particles of the decays or interactions, abbreviated as secondaries, are implemented.
They can either be extracted as pseudo particles just referring to the type of the interaction and its dynamic properties similar to the \texttt{ParticleState}.
Adaptations are made for continuous energy losses, which have an initial and final time or direction.
On the other side, also real particles can be sampled out of these pseudo decay or energy loss objects.

For the bremsstrahlung or the ionization, where just a photon or an electron is produced, this conversion is trivial.
For the electron or muon pair production, the asymmetry parameter $\rho$, see \eqref{eq:epair_dsigma} or \eqref{eq:mupair_dsigma} of the electron or muon pair can be sampled to distribute the energy loss among the individual particles.
For the photonuclear interaction, it is not possible in PROPOSAL to sample individual particles, since the complex calculation to simulate particles of a hadronic shower is a research topic on its own, still facing the problem of the \textit{muon puzzle} (see \secref{sec:air_shower}).
Existing tools are dealing specifically with hadronic cascades, like Sibyll \cite{Riehn20Sibyll} or QGSJet \cite{Ostapchenko10qgsjet}, which an applicant can use if this is required.
The direction of these secondaries of energy losses, either they are pseudo or real particles, are set to the direction of the primary particle before the stochastic process.
Although, there are deflection calculations for the primary particle, as described in \secref{sec:stochastic_deflect}, but no deflection for the secondaries, yet.

For the decay, there is also the possibility to sample the individual product particles out of a pseudo decay object.
However, the decay process differs compared to the energy loss processes, as not all of the decay energy is stored in an electromagnetic or hadronic cascade as some amount of the energy is taken away by the neutrinos.
Therefore, it is also possible to sample just the energy that is stored in a hadronic or electromagnetic cascade and therefore visible to the detector.
For the leptonic decays, only the produced electron or muon can be sampled.
For hadronic decays, only the neutrino energy needs to get sampled.

%

\subsection{Cross Sections}

Next to the cross-sections relevant for the muon propagation described in \secref{sec:interactions} there are further parametrizations available for most interactions to perform systematic studies on the effect of different cross-sections on the propagation.
The additional interactions and parametrizations for electrons, positrons, and high energy photons to produce an electromagnetic shower are based on the simulation software EGS \cite{EGS5} and are described in \cite{Alameddine20Master}.
For the neutrino propagation, the cross-sections of \cite{CSMS11NuXsection} are used which are already introduced for the weak interaction.
Additional effects like the Glashow resonance or neutrino oscillations are not implemented, yet.
For each of the main particle types that can be propagated, there is a default cross-section set available to reduce the complexity of the usage.
A complete description of the available processes and their parametrizations are listed in \tabref{tab:xsection_params}.
Their names are most often the authors of the corresponding publication, or with a descriptive naming of their purpose.
\begin{table}
    \caption{List of implemented cross-section parametrizations in PROPOSAL.}
    \label{tab:xsection_params}
    \begin{tabular}{l | l | l}
        \toprule
        \textbf{Ionization}          & \textbf{Photonuclear} & \textbf{Annihilation} \\
        BetheBlochRossi              & \underline{VMD}       & Heitler \\
        BergerSeltzerBhabha          & Zeus                  & \\
        BergerSeltzerMoller          & BezrukovBugaev        & \textbf{Compton} \\
                                     & Kokoulin              & KleinNishina \\
        \textbf{Bremsstrahlung}      & Rhode                 & \\
        KelnerKokoulinPetrukhin      & \underline{Regge}     & \textbf{Photo Pair Production} \\
        PetrukhinShestakov           & ALLM91\footnote{AbramowiczLevinLevyMaor91}                & Tsai \\
        CompleteScreening            & ALLM97\footnote{AbramowiczLevinLevyMaor97}                & \\
        AndreevBezrukovBugaev        & ButkevichMikheyev     & \textbf{$\mu$ Pair Production} \\
        SandrockSoedingreksoRhode    & RenoSarcevicSu        & KelnerKokoulinPetrukhin \\
        ElectronScreening            & AbtFT                 & \\
                                     & BlockDurandHa         & \textbf{Weak Interaction} \\
        \textbf{$e$ Pair Production} &                       & CooperSarkarMertsch \\
        KelnerKokoulinPetrukhin      & \underline{Shadowing} & \\
        SandrockSoedingreksoRhode    & DRSS\footnote{DuttaRenoSarcevicSeckel}                  & \\
        ForElectronPositron          & ButkevichMikheyev     & \\
        \bottomrule
    \end{tabular}
\end{table}

For \textbf{Ionization} the default cross-section for massive particles is the parametrization called \textit{BetheBlochRossi}, i.e. the parametrization described in \secref{sec:ioniz}.
Due to the additional Feynman diagrams for electrons with Bhabha scattering and positrons with M\o{}ller scattering, the default cross-section are slightly different and are labeled \textit{BergerSeltzerBhabha} and \textit{BergerSeltzerMoller} \cite{Bhabha36, Moller32, Berger64}.

For the \textbf{Bremsstrahlung}, the default cross-section is the widely used \textit{KelnerKokoulinPetrukhin} parametrization, described in \secref{sec:brems_screen_approx}.
A parametrization using the analytical interpolation of the screening as described in \eqref{eq:brems_screen_interpol} and another approach for the nuclear form factor is given by the \textit{PetrukhinShestakov} parametrization \cite{Petrukhin68}.
In the \textit{AndreevBezrukovBugaev} parametrization \cite{Andreev94Brems}, no approximation of the screening ($\Phi_1 \neq \Phi_2$) was made.
The calculation combining the benefits of these three parametrizations with additional radiative corrections is the \textit{SandrockSoedingreksoRhode} parametrization, which is described in \secref{sec:brems}.
There is also a parametrization approximated for high energies, called \textit{CompleteScreening}.
Since bremsstrahlung is the dominant energy loss process for electrons and positrons, there is a dedicated parametrization for them, called \textit{ElectronScreening}.

For the \textbf{$e$ pair production}, the widely used default is the \textit{KelnerKokoulinPetrukhin} parametrization \cite{Kokoulin71, Kelner98}, described in \secref{sec:epair_screen_approx}.
A parametrization without the screening approximation ($\Phi_1 \neq \Phi_2$) is \textit{SandrockSoedingreksoRhode} \cite{Soedingrekso19ICRC}.
There is a dedicated parametrization, \textit{ForElectronPositron}, again for electrons and positrons, due to the interference terms of the same particles in the final state.
This is similar to the \textbf{muon pair production} (c.f. \secref{sec:mupair}), where the parametrization is called \textit{KelnerKokoulinPetrukhin} \cite{Kelner00mupair}.

The \textbf{inelastic nuclear interaction} is the relevant processes with the highest theoretical uncertainties.
Hence the one with the most parametrizations that can be divided into the approach of a Vector Meson Dominance and the Regge models as described in \secref{sec:photonucl}.
For the VMD model, there are the parametrizations of \textit{BezrukovBugaev} \cite{Bezrukov80}, \textit{Kokoulin} \cite{Kokoulin97}, \textit{Rhode} \cite{Rhode93PhD}, and \textit{Zeus} \cite{Breitweg99ZEUS}, already described in \secref{sec:photo_vmd}.
For the Regge models, there is the initial \textit{AbramowiczLevinLevyMaor} parametrization published in the year 1991 \cite{Abramowicz91} and the parametrization from 1997 using updated fit parameters \cite{Abramowicz97}, where the latter is the default photonuclear parametrization.
This fit was redone on more recent HERA measurements in 2017 which can be used with \textit{AbtFT} \cite{Abt17PhotoQ2}.
Two alternative approaches can be used with the parametrization \textit{ButkevichMikheyev} \cite{Butkevich02} and with \textit{BlockDurandHa} \cite{Block14}.
Dedicated for supersymmetric staus, a calculation for spin 0 particles is available under the name \textit{RenoSarcevicSu}.
For the Regge models also the parametrization of the shadowing effect can be selected.
There is the \textit{ButkevichMikheyev} parametrization, corresponding to the same-named cross-section and alternatively the \textit{DuttaRenoSarcevicSeckel} \cite{Dutta01} parametrization of the shadowing effect.

For the dedicated interactions of recently added particles to propagate, i.e. electrons, positrons, high energy photons, and neutrinos, there is up to now, only one parametrization per interaction available.
The \textbf{Weak Interaction} either for neutrino propagation or for charged leptons is described by the parametrization of \textit{CooperSarkarMertsch} \cite{CSMS11NuXsection}.
For the \textbf{Annihilation} of a positron interacting with an atomic electron, the parametrization of \textit{Heitler} \cite{Heitler54} is used.
For high energy photons, the processes of \textbf{Pair Production} and \textbf{Compton Scattering} are included described by the parametrization names \textit{Tsai} and \textit{KleinNishina} respectively.

All cross-sections are implemented differential in the relative energy loss $v$ as $\sfrac{\dif \sigma}{\dif v}$, except for the purely stochastic processes, from now on called catastrophic interactions, where the primary particle does not exist anymore, i.e. Annihilation, weak interaction, and pair production by photons.
For cross-sections that are also differential in a second parameter, i.e. the asymmetry $\rho$ in the electron and muon pair production or the momentum $Q^2$ for the Regge approach of the photonuclear interaction, this second dimension is integrated numerically for a consistent treatment of the cross-sections.

To create a cross-section in PROPOSAL, one has to define at least a \texttt{ParticleDef} and a \texttt{Medium} or \texttt{Component} object, where the interaction takes place.
It is also possible to scale a cross section using a \texttt{Multiplier}, which can be adapted for each interaction individually.
Besides the implementation of different parametrizations for an interaction, this scaling can be used to analyze uncertainties of the cross section (c.f. \secref{sec:study}).
For the total cross-section or the average energy loss, also the energy loss cut, described in the next section %\sedref{sec:ecuts}
needs to be defined in case of non-catastrophic interactions.

%

\subsection{Energy Loss Cuts Separating Continuous and Stochastic Losses} \label{sec:ecuts}

Before describing the propagation principles the energy loss cuts, as an important mechanism to simulate muons, needs to be introduced.
As outlined in chapter \ref{sec:brems} the bremsstrahlung cross-section on an isolated atom diverges for small energy losses, meaning that there is an infinite possibility to create a photon with no energy.
Compared to the other interactions, where the lower limit is defined by the masses of the produced particles, the lower limit for bremsstrahlung is \num{0} due to the massless photon.
Although the bremsstrahlung cross-section does not diverge when propagating through media due to the dielectric effect, calculating this interaction is still numerically unstable and should be avoided.

Even when neglecting bremsstrahlung, it is also highly inefficient to simulate a huge amount of low energetic electrons produced in small energy losses regarding pair production or ionization with lower thresholds for the energy loss of $\approx\SI{1}{MeV}$ and $\mathcal{O}(\SI{100}{eV})$ respectively.
It would cost more runtime during the simulation and more resources to store all the energy losses, which might not even get measured by the detector.

Therefore all small energy losses up to a certain threshold are combined and averaged out into a continuous loss.
In PROPOSAL, this threshold can be set as an absolute energy, called $e_{\textrm{cut}}$ or relative to the energy of the primary particle, called $v_{\textrm{cut}}$.
The threshold is then chosen as the minimum between both,
\begin{align} \label{eq:ecut}
    e_{\textrm{cut}} = \min(e_{\textrm{cut}}', v_{\textrm{cut}} \cdot E_{\mathrm{particle}})
    \quad
    \text{with }
    e_{\textrm{cut}} \in (0, \infty)
    \text{ and }
    v_{\textrm{cut}} \in (0, 1] .
\end{align}
A purely continuous simulation can be achieved with $e_{\textrm{cut}}=\infty$ and $v_{\textrm{cut}}=1$.
This would be without any stochasticity and therefore deterministic.

First introduced in \cite{Lipari91}, the energy loss cut represents the threshold between an average energy loss
\begin{align} \label{eq:cuts_dedx}
    f(E) \coloneqq -\frac{\dif E}{\dif X} = 
        E \cdot \sum_\text{processes} \sum_{\substack{\text{atom} \\ \text{in medium}}}
        \frac{N_A}{A} \int_{v_{\min}}^{v_{\textrm{cut}}} v \frac{\dif \sigma}{\dif v} \dif v
\end{align}
and the number of stochastic losses per distance
\begin{align} \label{eq:cuts_dndx}
    \frac{\dif N}{\dif X} = 
        \sum_\text{processes} \sum_{\substack{\text{atom} \\ \text{in medium}}}
        \frac{N_A}{A} \sigma(E) ,
    \qquad \text{with} \qquad
    \sigma(E) \coloneqq \int_{v_{\text{cut}}}^{v_{\max}} \frac{\dif \sigma}{\dif v} \dif v .
\end{align}
The abbreviations $f(E)$ and $\sigma(E)$ are defined to be consistent with the literature.
With this approach, all small energy losses with an energy $E_{\text{Loss}} < e_{\text{cut}}$ are combined to a continuous loss and averaged out, while all energy losses above the cut are treated stochastically.
The energy loss cut is therefore an important parameter to adjust both the performance and precision of the propagation focusing on the former by using high cuts, or on the latter by using small cuts.
\begin{figure}
    \centering
    \begin{subfigure}[t]{0.47\textwidth}
        \centering
        \includegraphics[width=\textwidth]{./plots/dedx_ecut_1.pdf}
        \caption{$\left\langle \frac{\dif E}{\dif X} \right\rangle$ with $e_{\text{cut}} = \SI{1}{MeV}$.}
        \label{fig:dedx_ecut_1}
        \vspace{0.5cm}
    \end{subfigure}
    \hfill
    \begin{subfigure}[t]{0.47\textwidth}
        \centering
        \includegraphics[width=\textwidth]{./plots/dedx_ecut_500.pdf}
        \caption{$\left\langle \frac{\dif E}{\dif X} \right\rangle$ with $e_{\text{cut}} = \SI{500}{MeV}$.}
        \label{fig:dedx_ecut_500}
        \vspace{0.5cm}
    \end{subfigure}
    \begin{subfigure}[t]{0.47\textwidth}
        \centering
        \includegraphics[width=\textwidth]{./plots/dndx_ecut_1.pdf}
        \caption{$\frac{\dif N}{\dif X}$ with $e_{\text{cut}} = \SI{1}{MeV}$.}
        \label{fig:dndx_ecut_1}
    \end{subfigure}
    \hfill
    \begin{subfigure}[t]{0.47\textwidth}
        \centering
        \includegraphics[width=\textwidth]{./plots/dndx_ecut_500.pdf}
        \caption{$\frac{\dif N}{\dif X}$ with $e_{\text{cut}} = \SI{500}{MeV}$.}
        \label{fig:dndx_ecut_500}
        \vspace{0.5cm}
    \end{subfigure}
    \caption{Continuous energy loss and stochastic interaction probability of muons in Ice for an energy loss cut of \SI{1}{MeV} and \SI{500}{MeV}.}
    \label{fig:dedx_dndx_ecuts}
\end{figure}

In \figref{fig:dedx_dndx_ecuts} the effect of different energy loss cuts on the average energy loss and the stochastic interaction probability is shown.
Compared to the average energy loss without an energy loss cut (see \figref{fig:dedx_all}), the ionization dominates the small energy losses, also at higher muon energies.
The huge amount of low energetic ionization losses is also seen in the stochastic interaction probability and the rise of the number of secondaries between the already low cut of \SI{500}{MeV} and the cut at \SI{1}{MeV}.

The integrals presented in \eqref{eq:cuts_dedx} and \eqref{eq:cuts_dndx} are interpolated and can then be used for the propagation integrals described in the next section.
Next to these two integrals of the cross-sections, a third integral is created when using the option \texttt{continuous\_randomization}, described in section \ref{sec:cont_rand}.
Hereby, the second momentum of the cross-section
\begin{align} \label{eq:ecut_de2dx}
    \left\langle \frac{\dif E^2}{\dif X} \right\rangle = E^2 \int\limits_{v_{\min}}^{v_{\textrm{cut}}} v^2 \frac{\dif \sigma}{\dif v} \dif v
\end{align}
is calculated and used to smear out the continuous energy loss and slightly randomize the deterministic calculation.

%

\subsection{\textit{One propagation step for PROPOSAL}} \label{sec:prop_util}
% \textit{\dots a giant leap for muon research ;).}

Out of the cross-sections and the energy loss cuts, the propagation integrals, collected in the so-called \texttt{propagation\_utilities} can be calculated.
These integrals or utilities are required to perform a single propagation step.

There are two approaches on how PROPOSAL can estimate or sample from an initial state $i$ the next interaction point with the final state $f$;
\begin{itemize}
    \item by solving the energy integral taking into account the continuous losses,
    \item and by using the mean free path length assuming no continuous losses.
\end{itemize}
If the particle propagation includes also continuous energy losses, which is the case for charged leptons, the default approach to sample the next interaction is by solving the so-called \textbf{Energy Integral}
\begin{align} \label{eq:prop_cdf_general}
    P(E_i \leq E \leq E_f) = - \int\limits_{E_i}^{E_f} p(E) \dif E ,
\end{align}
with the probability distribution $p(E)$ for an interaction and the cumulative distribution $P(E)$.
The general idea was developed in \cite{Sokalski01MUM} and \cite{Chirkin04MMC}.
Instead of sampling a distance or length, the calculation is performed in energies, which is more accurate and numerically stable and e.g. independent of the density (when neglecting effect such as the LPM dependence).
The goal is to sample the energy $E_f$ the muon has, right before it has the next stochastic interaction while the difference between the initial energy $E_i$ and $E_f$ is lost due to continuous losses.
In a second step, the distance is calculated according to the sampled $E_f$, see \eqref{eq:tracking_integral}.

To derive the energy integral to calculate the next stochastic point, the track between the initial and final state is divided into infinitesimal small track sections $\Delta x$.
The probability of having no stochastic loss in each of these track sections but one stochastic interaction at the state $f$ can be written as
\begin{align}
    \Delta P_f &= \prod\limits_{j=i}^{f-1} (1 - \sigma_j \Delta x_j) \sigma_f \Delta x_f \\
        &\approx \exp \left( - \sum\limits_{j=i}^{f-1} \sigma_j \Delta x_j \right) \sigma_f \Delta x_f \\
    \xrightarrow{\Delta x \to 0} \dif P_f &= \exp \left( - \int\limits_{x_i}^{x_f} \sigma(E(x)) \dif x \right) \sigma_f \dif x_f \label{eq:prop_interaction_integral}
\end{align}
where the approximation of $\Delta x \ll 1$ is used in the second line and $\sigma_j = \sigma(E(x_j))$ is the probability for a stochastic loss at state $j$ as defined in \eqref{eq:cuts_dndx}.
For the transformation to an energy integral, \eqref{eq:cuts_dedx} is used, resulting in
\begin{align}
    \dif P_f = \exp \left( \int\limits_{E_i}^{E_f} \frac{\sigma(E)}{f(E)} \right) \frac{\sigma(E_f)}{-f(E_f)} \dif E_f .
\end{align}
When integrating over the probabilities the cumulative distribution of \eqref{eq:prop_cdf_general} is obtained
\begin{align}
    P(E_f \leq E \leq E_i) = \int\limits_{P_i=0}^{P_f} \dif P_f .
\end{align}
This can be integrated using the substitution
\begin{align}
    u(E) = \int_{E_i}^E \frac{\sigma(E')}{f(E')} \dif E'
    \qquad \text{and} \qquad
    \dif u = \frac{\sigma(E)}{f(E)} \dif E ,
\end{align}
resulting in
\begin{align}
    P(E_f \leq E \leq E_i) = \exp \left( \int\limits_{E_i}^{E_f} \frac{\sigma(E)}{f(E)} \dif E \right) .
\end{align}
Finally, the energy $E_f$ can be sampled using a random number $\xi \in (0,1]$ with
\begin{align} \label{eq:energy_integral_int}
    \logn \xi = \int\limits_{E_i}^{E_f} \frac{\sigma(E)}{f(E)} \dif E ,
\end{align}
which has a solution if
\begin{align}
    \xi < \exp \left( \int\limits_{E_i}^{E_{\text{low}}} \frac{\sigma(E)}{f(E)} \dif E \right) ,
\end{align}
where $E_{\text{low}}$ is the \texttt{low} parameter of the particle definition, described in section \ref{sec:particle}, and therefore always smaller than $E_i$.
If the random number is greater than the integral, there is no stochastic loss.

If the particle can decay, the energy $E_{f,\text{decay}}$, where the next decay would occur, can be sampled similarly to solving the energy integral for interactions.
By replacing the interaction probability in \eqref{eq:energy_integral_int} with the decay cross-section defined in \eqref{eq:sigma_decay}, the \textbf{Decay Integral} is defined by
\begin{align}
    \rho \logn \xi = \int\limits_{E_i}^{E_{f,\text{decay}}} \frac{\dif E}{f(E) \gamma\beta \tau c_0} .
\end{align}
In contrast to the interaction cross-sections, the decay cross-section is independent of the medium and the density does not cancel out with the continuous losses.
Therefore, the density must be taken into account.

Out of the sampled energy $E_f$, the distance where the next stochastic loss would occur can be calculated using the so-called \textbf{Tracking Integral}, defined by
\begin{align} \label{eq:tracking_integral}
    - \int\limits_{E_i}^{E_f} \frac{\dif E}{f(E)}
    = \int\limits_{x_f}^{x_i} \rho(x) \dif x
    \xrightarrow{\rho=const.}
    x_i - x_f .
\end{align}
For inhomogeneous media, the density profile needs to be considered for the tracking integral.
Alternatively, the distance can be calculated in units of grammage instead of distances, separating the density distribution from the tracking integral.

Regarding purely stochastic propagations, e.g. for neutrinos, there is no energy loss between two stochastic interactions.
Therefore the calculations are not in energies, but in distances, and the interaction and tracking integrals \eqref{eq:prop_interaction_integral} and \eqref{eq:tracking_integral} reduce to
\begin{align} \label{eq:tracking_stochastic}
    x_f - x_i = \frac{- \logn \xi}{\sigma(E)} .
\end{align}
This equation can also be derived by sampling from an exponential distribution with the mean free path length of the interactions, which is proportional to the inverse cross-section.
It is also applicable for scenarios where continuous energy losses are present, but where the step length between two stochastic losses is small.
In \figref{fig:prop_util_tracking}, a comparison between the two approaches of calculating the next interaction point is shown.
Only for the rare scenarios, in which most of the energy is lost, there are larger deviations.
\begin{figure}
    \centering
    \includegraphics[width=0.82\textwidth]{./plots/prop_util_next_int.pdf}
    \caption{Calculation of the distance to the next interaction point for a muon with an initial energy of \SI{2e5}{MeV} in ice using an energy loss cut of \SI{500}{MeV}. The approach using the tracking integral via the sampled energy integral is compared to a sampling of an exponential probability distribution function (pdf) with the mean free path length as the scale parameter.}
    \label{fig:prop_util_tracking}
\end{figure}

Next to the sampled decay or interaction energy, also energy thresholds can be set optionally limiting estimation of the next track point or energy.
This can either be a limitation in the maximum energy the particle can lose continuously between two stochastic interactions.
Another limitation of the energy is that the particle energy threshold has been reached, i.e. the minimal energy, until the particle should get propagated.
The final energy of a single propagation step is therefore defined by
\begin{align} \label{eq:prop_energy_choose}
    E_f = \max(E_{\text{interaction}}, E_{\text{decay}}, E_i - E_{\text{max continuous loss}}, E_{\text{min particle}})
\end{align}
% The step length can also be limited optionally, which is considered after calculating the tracking integral.
The next track point is finally chosen between the sampled interaction point, calculated either via the estimated energy with \eqref{eq:prop_energy_choose} or directly via \eqref{eq:tracking_stochastic}, the step limitation, and the edge of the current geometry, the particle is propagated through
\begin{align}
    x_f = \min(x_{\text{interaction}}, x_{\text{max step}}, x_{\text{border}}) .
\end{align}
The limitation of the step length, in the energy or the distance, is important for processes assuming a constant particle energy between two stochastic interactions, especially when calculating the magnetic pulse of an air shower.
However, when forcing maximum continuous losses of e.g. \SI{1}{\%} of the initial energy, this can results in many small propagation steps without a stochastic loss for lower energies and thereby an inefficient performance.

The \textbf{Time Integral} calculating the time until the particle reached a certain energy loss can either be calculated in a similar way as the tracking integral including the continuous losses resulting in
\begin{align} \label{eq:time_integral}
    t_i - t_f = \int\limits_{E_i}^{E_f} \frac{\dif E}{f(E)} \underbrace{\frac{1}{c_0 \sfrac{p}{E}}}_{v(E)} .
\end{align}
For only stochastic propagations of massless particles like photons or neutrinos, the time can be calculated with
\begin{align}
    t_i - t_f = \frac{x_f - x_i}{c_0}.
\end{align}
At high energies, this is also an accurate approximation if the particles are massive and lose energy along the track since the assumption of propagating with the speed of light is a good approximation, as shown in \figref{fig:prop_util_time}.
\begin{figure}
    \centering
    \includegraphics[width=0.82\textwidth]{./plots/prop_util_next_time.pdf}
    \caption{Calculation of the time at the next interaction point for a muon with an initial energy of \SI{e5}{MeV} in ice using an energy loss cut of \SI{500}{MeV}. The approach of the time integral is compared to the approximation that the particle propagates with $c_0$.}
    \label{fig:prop_util_time}
\end{figure}

In case a non-catastrophic interaction is chosen, the relative energy loss of the stochastic interaction is sampled with the so-called \textbf{Stochastic Integral}
\begin{align}
    \xi = \frac{1}{\sigma(E)} \int\limits_{v_{\text{cut}}}^{v_{\text{Loss}}} \frac{\dif \sigma}{\dif v} \dif v .
\end{align}
The estimation of the stochastic interaction can be divided into three parts; the interaction type, the target with which it interacts, i.e. the atom or in case of ionization the medium, and the relative energy loss.
By stacking these probabilities, the interaction can be sampled with a single inverted cumulative probability distribution, shown in \figref{fig:prop_util_stoch_loss}.
\begin{figure}
    \centering
    \includegraphics[width=0.82\textwidth]{./plots/prop_util_stoch_loss.pdf}
    \caption{The stacked and inverted cumulative distribution of a stochastic loss for a muon with an initial energy of \SI{e5}{MeV} in ice using an energy loss cut of \SI{500}{MeV}. The interaction is split between the probabilities for the different targets; Hydrogen (dotted), Oxygen (dashed), and Ice (dash-dotted).}
    \label{fig:prop_util_stoch_loss}
\end{figure}

After calculating the stochastic energy loss, the stochastic deflection can optionally be sampled, as described in section \ref{sec:stochastic_deflect}.
Finally, the remaining dynamic properties (see \secref{sec:particle}) of the particle like the propagated distance are updated.
If the particle didn't lose too much energy that its remaining energy is still above the threshold, the cycle starts again and the next interaction point is sampled.

These are the main calculations, which are performed in each step of the particle propagation.
They can optionally further be improved via the two processes described in the following section; the so-called \textit{Continuous randomization} and \textit{Multiple Scattering}.

\subsection{Continuous Randomization} \label{sec:cont_rand}

As already described in section \ref{sec:ecuts}, the choice of the energy loss cut mainly influences the performance of the simulation.
Smaller cuts are more accurate, thus slower and higher cuts are more performant, thus less precise. 
In any case, this is an unphysical cut producing an artifact in the simulation.
Therefore the cut has to be chosen, that these artifacts are not visible in the simulation of the experiment, or at least they should be reduced as much as possible.

There are two main scenarios with different requirements for muon simulations.
Calorimetric measurements are often more sensitive to the energy losses than to the bare muon track.
An absolute value of the energy loss cut is then often used according to the detection sensitivity.
The other scenario is propagating muons through large distances to the detector, where the track of the muon with its energy losses is not visible to the detector.
This is often the case for experiments located deep underground.
Here, only an accurate description of the incoming muon flux at the detector is required and not a precise calculation of the energy losses.
Therefore, a relative energy loss cut is often used in these cases.
Regarding the IceCube detector, both scenarios are required.

When setting the absolute energy of the cut below the detection sensitivity of the calorimeter, the artifacts of the energy loss cut are not visible to the detector.
However, for the second scenario with the performant muon simulation and a relative cut, detectable artifacts may remain in the muon flux.
\begin{figure}
    \centering
    \begin{subfigure}{0.9\textwidth}
        \centering
        \includegraphics[width=\textwidth]{./plots/prop_cont_rand_loss.pdf}
        \caption{The spectrum of the lost energy.}
        \label{fig:cont_rand_loss}
        \vspace{0.5cm}
    \end{subfigure}
    \begin{subfigure}{0.9\textwidth}
        \centering
        \includegraphics[width=\textwidth]{./plots/prop_cont_rand_mu.pdf}
        \caption{The muon energy spectrum after propagation.}
        \label{fig:cont_rand_mu}
    \end{subfigure}
    \caption{The energy spectrum of the muon energy and the summed energy lost after propagating \num{e6} muons through \SI{1}{m} of ice. For different energy loss cuts $v_{\text{cut}}$ the energy spectra are compared including continuous randomization (dashed) and without this smearing (solid).}
    \label{fig:cont_rand}
\end{figure}

The main artifact can be seen in the energy spectrum of the muons after propagating a certain distance, as shown in \figref{fig:cont_rand_mu}.
Each muon propagating the distance without any stochastic interaction arrives with the same energy, resulting in a peak in the distribution.
The position of the peak is determined by the continuous losses of the muon according to the cut.
After the peak, there is a gap in the spectrum of the size of the energy loss cut, since a single stochastic loss requires at least the amount of the energy loss cut.
\begin{table}
    \caption{Comparison of the runtime performance of \num{e6} muons propagated with different energy loss cuts through \SI{1}{m} of ice.}
    \label{tab:cont_rand_runtime}
    \begin{center}
    \begin{tabular}{l | c | c | c | c }
        \toprule
        Continuous & \multicolumn{4}{c}{Energy Loss Cut $v_{\mathrm{cut}}$} \\
        Randomization & \num{e-2} & \num{e-3} & \num{e-4} & \num{e-5} \\
        \midrule
        True & \SI{9}{s} & \SI{12}{s} & \SI{21}{s} & \SI{93}{s} \\
        False & \SI{9}{s} & \SI{12}{s} & \SI{20}{s} & \SI{89}{s} \\
        \bottomrule
    \end{tabular}
    \end{center}
\end{table}

The main idea is now to smear out the sampled energy $E_f$ of the muon, after the calculation of the tracking and the time integral to randomize the continuous propagation step.
The randomization is performed using a Gaussian distribution with the mean $E_f$.
The variance of the distribution is calculated using the second momentum of the energy loss, as already indicated in \eqref{eq:ecut_de2dx}.
The variance, defined by
\begin{align}
    \left\langle \frac{\Delta (\Delta E)^2}{\Delta x} \right\rangle &= \left\langle \frac{\Delta E^2}{\Delta x} \right\rangle - \left\langle \frac{\Delta E}{\Delta x} \right\rangle^2
\end{align}
can be calculated similar to the derivation of the energy integral, dividing the track between two stochastic losses in many small track segments and summing up their contribution
\begin{align}
    \langle \Delta (\Delta E)^2 \rangle
        &= \sum\limits_{j=i}^f \left[ \left\langle \frac{\Delta E^2}{\Delta x} \right\rangle_j \Delta x_j
            - \left\langle \frac{\Delta E}{\Delta x} \right\rangle_j^2 \underbrace{(\Delta x_j)^2}_{\approx 0} \right]
        \\
    \xrightarrow{\Delta x \to 0} &\approx \int\limits_{x_i}^{x_f} \left\langle \frac{\dif E^2}{\dif x} \right\rangle \dif x.
\end{align}
In the second line, the limit of infinitesimally small track lengths is used neglecting the terms of $(\Delta x)^2$.
The integral over the track segments can again be written as an integral over the energies, similar to the energy integral for the interaction or decay, resulting in the \textbf{Continuous Randomization Integral}
\begin{align}
    \langle \Delta (\Delta E)^2 \rangle = \int\limits_{E_i}^{E_f} \frac{E^2}{-f(E)} \left\langle \frac{\dif E^2}{\dif x} \right\rangle .
\end{align}
The effect of the runtime for the \num{e6} muons propagated for \figref{fig:cont_rand} is listed in \tabref{tab:cont_rand_runtime}.
A reduction in runtime performance is visible if the continuous randomization is included.
However, while for high energy cuts, other processes are more dominant and the different cuts do not influence significantly the runtime, this changes drastically for smaller cuts increasing the runtime by nearly an order of magnitude due to the additional propagation steps.

%

\subsection{Multiple Scattering}

The theory of multiple scattering of a muon between two stochastic losses has already been described in section \ref{sec:multiple_scat}.
There are several options available to calculate the multiple scattering:
\begin{itemize}
    \item It is possible to propagate without scattering to increase the performance if the deflections are not relevant.
    \item The other extreme is a precise calculation of \texttt{Moliere}'s theory on multiple scattering, described in detail in \cite{Geiselbrinck13Bachelor}.
    This however can increases the runtime performance by orders of magnitude depending on the energy, as presented in \cite{Dunsch19PROPOSAL}.
    \item The \texttt{Highland} approximation to Moli\`{e}re's theory using a Gaussian distribution, as described in \eqref{eq:multiple_scat}.
    \item The Highland approximation, as described before, but considering also the continuous energy losses, is called \texttt{HighlandIntegral}.
\end{itemize}
Only the latter includes the continuous energy loss during the propagation step, while the others assume a constant particle energy of $E_i$.
Including the continuous loss, the calculation of the scattering angle in the Highland approximation given by \eqref{eq:multiple_scat} changes to the \textbf{Scattering Integral}
\begin{align}
    \theta_0 = \SI{13.6}{MeV} \left( 1 + 0.088 \log_{10} \frac{X}{X_0} \right)
        \sqrt{ \int_{E_f}^{E_i} \dif E \frac{E^2}{p^4} \frac{1}{-f(E) X_0} } .
\end{align}
\begin{figure}
    \centering
    \includegraphics[width=0.9\textwidth]{./plots/prop_scat.pdf}
    \caption{Comparison of different deflection calculations for \num{e6} muons propagating \SI{1}{km} through ice with an energy loss cut of \SI{500}{MeV}. Either only multiple scattering according to the parametrization is used, or only stochastic deflections for the scattering, or both processes. In the latter case, the HighlandIntegral parametrization for multiple scattering is used.}
    \label{fig:prop_scat}
\end{figure}
\begin{table}
    \caption{Comparison of runtime performance of \num{e6} muons propagated with different Scattering calculations through \SI{1}{km} of ice with an energy loss cut of \SI{500}{MeV}.}
    \label{tab:scatter_runtime}
    \begin{center}
    \begin{tabular}{l | c }
        \toprule
        Scattering Mode & Runtime / s \\
        \midrule
        No Scattering & 404 \\
        Moliere & 894 \\
        Highland & 416 \\
        HighlandIntegral & 463 \\
        Stochastic Deflection & 428 \\
        Cont. and Stochastic Deflection & 439 \\
        \bottomrule
    \end{tabular}
    \end{center}
\end{table}

In \figref{fig:prop_scat} the effect of the different parametrizations for multiple scattering on the muon simulation is shown and compared also to the effect of stochastic deflections.
As already discussed in \secref{sec:multiple_scat}, the Highland parametrization is an accurate approximation for small angles but does not accurately describe the tail in the distribution of the Moli\`{e}re scattering at high scattering angles.
Including the continuous energy losses for the scattering angle has only an influence when using higher energy loss cuts and larger steps.
Thereby, the runtime increased by a couple of percents when including a scattering calculation, as shown in \tabref{tab:scatter_runtime}.
Only the calculation of the Moli\`{e}re scattering drastically increases the performance by more than a factor of 2, which is even larger for higher energetic particles.

For the stochastic deflection, there are also multiple parametrizations available in PROPOSAL.
The standard parametrizations for the muon deflection are described in section \ref{sec:stochastic_deflect} and discussed in detail in \cite{Gutjahr21Master}.
They only have minor influences on most interactions with small deflection angles.
However, for larger deviations of the muon axis, they contribute significantly to the distribution, and in the tail, their influence is comparable to the effects of the Moli\`{e}re scattering.

%

\subsection{The Propagator}

The modules described above are combined to finally create the so-called Propagator, which propagates the particle until a certain condition and returns the track.
The propagator can be initialized using the following steps:
\begin{enumerate}
    \item Define a particle definition, a medium, a selection of cross-section parametrizations, and the energy loss cuts, if necessary, to create the cross-section integrals.
    \item Out of the cross-section integrals, the propagation utilities can be defined. The utilities consist of the following modules:
    \begin{itemize}
        \item An interaction module providing the energy integral and the stochastic integral.
        \item A displacement module providing the tracking integral and the calculator of the mean free path length.
        \item A time module, calculating the time either with the time integral or with the approximation assuming a velocity of $c_0$.
        \item An optional decay module providing the energy integral for the decay process.
        \item An optional continuous randomization module.
        \item An optional scattering module containing the multiple scattering and the stochastic deflections.
    \end{itemize}
    \item The propagation utilities, together with a geometry and a density profile, define a sector, where a particle can propagate through.
    \item With a list of sectors, e.g. differing in the energy loss cuts before and inside the detector, the propagator can be initialized.
\end{enumerate}
These objects can either be defined explicitly in a script using the PROPOSAL library, or these settings can be defined inside a \texttt{json} configuration file.

After the initialization of the settings, the propagator can propagate a particle.
The propagation loop starts with the sampling of the next interaction point or energy.
After that, further parameters of the final particle state like the time are estimated including optional corrections due to scattering or the energy randomization.
Then, the stochastic loss is sampled and the cycle of propagation starts again until it terminates.
There are the following termination conditions for the propagation:
\begin{itemize}
    \item The particle decays.
    \item The particle has a catastrophic interaction, e.g. weak interaction or annihilation, and ceases to exist after the interaction.
    \item The particle has reached the end of the defined environment and there is no further sector geometry in the direction of the particle.
    \item An optionally set maximum propagation length has been reached.
    \item An optional minimum of the particle energy has been reached.
    \item The particle leaves the detection region.
\end{itemize}
Regarding the latter case, it is of interest for experiments on how the muons propagate before they reach the detector and in particular how they behave inside the detector.
But when leaving the detection volume, muons do not need to get propagated further.
They might still be highly energetic and propagate large distances e.g. a neutrino-induced, up-going muon in IceCube can propagate to the stratosphere and beyond.
Therefore a threshold of the hierarchy in the geometries is implemented.
Each sector defining the detection area contains a hierarchy above this threshold.
If the particle enters a sector above this hierarchy threshold, it propagates until it reaches the border of all sectors above the threshold.
If the next sector has a hierarchy below the threshold, the propagation stops.

After the propagation, the track of the particle is returned consisting of the interaction points and the entry and exit points of geometries.
Out of this track, the continuous or the stochastic losses can be extracted and filtered for a specific interaction type.
Also, the secondary particles of the interactions or the decay can be produced as described in \secref{sec:particle}.

Using the interaction points in the track, also the particle state at each point of the continuous step can be extracted using re-simulations.
For a given energy, the tracking integral \eqref{eq:tracking_integral} and the time integral \eqref{eq:time_integral} are used to determine the particle state.
Alternatively, a propagated distance can be given to determine the particle state at this distance.
This deterministic approach of re-propagating one step is not useful when including continuous randomization, and will produce inconsistent results due to the shift of the final energy.
Also, the multiple scattering is just approximated with a straight line between the initial and the final state.
In principle, the particle would scatter less at high energies in the beginning and deviate more in the latter part of the track at lower energies.
However, since the random state is not stored for each step, a straight line is the most generic approximation.
The particle track of a single muon including its energy losses is shown in \figref{fig:prop_track}.
\begin{figure}[H]
    \centering
    \includegraphics[width=0.82\textwidth]{./plots/prop_track.pdf}
    \caption{The energy loss of a \SI{10}{TeV} muon during its propagation through ice with an energy loss cut of \SI{500}{MeV} until it decays is shown in the upper plot. The deflection of the above-mentioned muon projected on the $x$ axis, while initially propagated along the $z$-axis, including the energy losses on the track is shown in the lower plot. The radius of the energy loss circles is scaled with the square root of the energy loss.}
    \label{fig:prop_track}.
\end{figure}

%

\subsection{Decay}

For the different decay channels of muons and taus, multiple decay sampling methods are implemented in PROPOSAL.
The two leptonic decay methods with an approximated production of the electronic decay channel and the more accurate approach for the muonic decay channel are already described in \secref{sec:decay_dsigma}.

Regarding the hadronic decay modes of the tau, there are two methods available, both only considering the phase space sampling, not the matrix element.
In the first versions of PROPOSAL, there was only the phase space sampling for a two-body decay implemented.
This is exact for the decay into a Pion and a neutrino.
The decay channels where more decay products are produced were described effectively with a two-body decay into an intermediate particle, a resonance, that predominantly decays into the desired products.
For example, the decay channel $\tau^- \to \pi^- \pi^0 \nu_\tau$, which is the largest decay channel of a tau, was described as a two-body decay into the neutrino and $\rho(770)^-$, which decays with more than \SI{99}{\%} in the channel  $\rho(770)^- \to \pi^- \pi^0$.
Therefore, it is a reasonable approximation for a simple description of the tau decay.
However, without further dynamics of a matrix element, a pure phase space consideration of a two-body decay results in a constant, deterministic value of the produced particle energies in the rest frame of the tau.
This can be explained since for the two particles in the rest frame of the primary, there is only one possible configuration of one particle going in one direction and the other particle in the opposite direction.
In the resulting energy distribution of the hadronic particles in the rest frame, four peaks occurred for the four implemented decay channels.
In the boosted frame of the primary particle, this resulted in step functions in the energy distribution of the hadronic products, with a step at the mass of a hadronic product.
This is further explained in \cite{Dunsch19PROPOSAL} visualizing also the artifacts in the energy spectra.

Since these step functions were observed in IceCube simulations, a many-body phase space sampling was introduced in \cite{Dunsch18Master}.
Thereby, the Raubold-Lynch algorithm is used, recursively factorizing an $n$-body decay into $n$ two-body decays, which can be calculated.
Since this algorithm generates decay events, which are not equally distributed in the phase space, a rejection sampling can be applied to extract a uniformly distributed phase space.
However, since the generation of a single decay product configuration already requires $3n - 4$ random numbers, the rejection sampling can increase the amount of required random numbers by an order of magnitude depending on the configuration.
If the requirement on the decay simulation is only a continuous spectrum without steps, this uniform sampling is not necessary.
Regarding the runtime, the decay calculation is not critical compared to the propagation and a more accurate sampling of the phase space doesn't change this relation.
But the huge number of random numbers, which can be in the order of $\mathcal{O}(\num{e2})$ for a single decay calculation due to the simple rejection sampling, needs to be considered if an efficient usage of random numbers is necessary.
This could be improved in future works by introducing more efficient methods like importance sampling.

Yet, there is no matrix element implemented for any decay channel and a constant matrix element of 1 is used.
However, it is possible to include an external function calculating the matrix element.
Thereby, the aforementioned rejection sampling for a uniform phase space gets weighted according to the matrix element.
This is also applicable and can be tested for the muon decay, where the matrix element is defined by
\begin{align}
    \mathcal{M} = 64 G_F^2 p_1 p_2 ,
\end{align}
with
\begin{align}
    p_1 = E_\mu E_{\bar{\nu}_e} - \vec{p}_\mu \cdot \vec{p}_{\bar{\nu}_e}
    \qquad \text{and} \qquad
    p_2 = E_e E_{\nu_\mu} - \vec{p}_e \cdot \vec{p}_{\nu_\mu} .
\end{align}
Comparing the sampled electron spectra in \figref{fig:decay_sampling_mu_products} with the two leptonic decay sampling methods shows, that the many-body phase space sampling including the matrix element is as accurate as the improved leptonic decay sampling.
\begin{figure}
    \centering
    \begin{subfigure}{0.8\textwidth}
        \centering
        \includegraphics[width=\textwidth]{./plots/decay_TauMinus_decay_spectrum}
        \caption{Energy spectra of all decay products.}
        \label{fig:decay_sampling_all_products}
        \vspace{0.5cm}
    \end{subfigure}
    \begin{subfigure}{0.8\textwidth}
        \centering
        \includegraphics[width=\textwidth]{./plots/decay_MuMinus_spectrum_of_TauMinus_decay}
        \caption{Energy spectrum of the produced muons.}
        \label{fig:decay_sampling_mu_products}
    \end{subfigure}
    \caption{The Energy Spectra of the decay product of \num{e7} taus decaying at rest. The approximated (PDG) and the more accurate (LahiriPal) leptonic decay calculations are compared to the many-body decay calculation.}
    \label{fig:decay_sampling}
\end{figure}
However, regarding also the sampled energies of the two neutrinos in \figref{fig:decay_sampling_all_products}, different shapes of the spectra are observed.
In the two leptonic decay sampling approaches only for the charged, massive leptonic product, the energy, and the direction get sampled.
For the neutrino states, only the direction for one neutrino gets sampled while the other neutrino gets the opposite direction, in the rest frame.
The energy for both neutrinos is also not sampled and each neutrino receives the energy
\begin{align}
    E_\nu = \frac12 \sqrt{(m_\mu^2 - E_e)^2 - p_e^2} .
\end{align}
On the other side, the many-body phase space calculation with the matrix element is assumed to produced the more accurate result.
Comparing the neutrino spectra in \figref{fig:decay_sampling_all_products} with \figref{fig:decay_spectrum_compare} indicates that the many-body decay calculation produces a more plausible spectrum of neutrino energies.
Since the neutrino spectra have not been validated, yet, further investigations are necessary, especially in the light of a growing interest in simulating tau regeneration through the earth.
Therefore, also a more accurate description of the hadronic decay products is necessary.
But as already mentioned, the processes of hadronic interactions are a research topic on its own with dedicated tools for each problem.
For the hadronic tau decay, the default simulation framework is TAUOLA \cite{Jadach91, Jadach93, Davidson12, Chrzaszcz16}, used in nearly every experiment requiring an accurate description of the tau decay.
In a future extension of PROPOSAL, an interface to this framework can be implemented.

%
% PROPOSAL
%

\section{Usage of PROPOSAL in Simulation frameworks}

Initially intended as a muon propagator, PROPOSAL evolves to an electroweak interaction module.
PROPOSAL is currently used in many different applications, from large simulation frameworks to small case studies.
Since it is written in C\texttt{++} and also callable in python, it is easily adaptable in new applications.
It has common installation approaches as a classic \texttt{cmake} project, which is used in most scientific software frameworks, or using the conan package manager.
The dependencies of \texttt{boost} and \texttt{eigen}, which are already available in most scientific frameworks as well as the widely used \texttt{spdlog} are also \texttt{cmake} projects.
To build the python interface, the widely used library connecting C\texttt{++} and python \texttt{pybind11} is used.
With the simple installation via pip (\texttt{pip install proposal}), PROPOSAL is now also used in many small-scaled simulation studies.

%

\subsection{High Energy Neutrino Telescopes}

High energy neutrino telescopes, especially the IceCube experiment, were the initial purpose and are still the main users of PROPOSAL.
For the IceCube simulation, PROPOSAL propagates high energy muons and taus as e.g. described in \cite{IceCube2016Aachen}.
The IceCube collaboration is also providing significant contributions to the development.
Also in the simulation chains of other neutrino telescopes PROPOSAL is available for their muon propagation, like in the Baikal experiment \cite{Pastircak19Baikal} or KM3NeT \cite{Km3Net2020gSeaGen}.

Inside the IceCube simulation, PROPOSAL is used with two configurations.
The generated muon events from atmospheric air showers, which CORSIKA can propagate to the surface of the ice, get further propagated through the ice to the detector region with PROPOSAL.
The neutrino-induced muons are generated inside the ice (not necessarily directly at the detector) or the bedrock below the detector, using e.g. GENIE \cite{Andreopoulos10genie} or ANIS \cite{Gazizov05anis} or the recently developed LeptonInjector \cite{Icecube21LeptinInjector}, and then get further propagated with PROPOSAL.
Since these muons can have high energies propagating tens of kilometers through the ice, as shown in \figref{fig:prop_range}, a relatively high energy loss cut of $v_{\mathrm{cut}} = \num{e-2}$ is used with the continuous randomization option.
Thereby, only the final muon state at the detector entrance is of interest, as well as stochastic interactions, where again long-ranged muons are produced, that can reach the detector.
These interactions are photonuclear interactions, $\mu^+\mu^-$ pair production, and weak interactions.

The second configuration is the propagation inside the detector with an energy loss cut of \SI{500}{MeV}, since the detector is not sensitive to smaller energy losses.
An energy distribution of the produced secondaries inside the detector is shown in \figref{fig:prop_sec_dist}.
After the propagation with PROPOSAL, the next module in the simulation chain, the Cascade Monte-Carlo (CMC),uses the the stochastic interactions along the muon track and samples the Cherenkov photons according to the energy and differentiating between electromagnetic and hadronic cascades.
Also, the Cherenkov photons along the continuous energy loss step of the muon is simulated according to an energy loss cut of \SI{500}{MeV}.
% Changing this cut would also require creating new photon tables with GEANT4.
\begin{figure}[H]
    \centering
    \includegraphics[width=0.82\textwidth]{./plots/ranges_prop.pdf}
    \caption{The energy dependent range distribution of muons in ice. For each energy bin, \num{e3} muons are propagated with a relative energy cut $v_{\mathrm{cut}}=\num{e-2}$. The median in each energy bin is compared to the Fit of the average energy loss, described in \secref{sec:dedx}.}
    \label{fig:prop_range}.
\end{figure}
\begin{figure}
    \centering
    \includegraphics[width=0.82\textwidth]{./plots/prop_secondary_dist.pdf}
    \caption{The energy distribution of the secondaries produced when propagating \num{e6} muons through \SI{1}{km} of ice using an energy loss cut of \SI{500}{MeV}.}
    \label{fig:prop_sec_dist}.
\end{figure}

Next to the muon, also tau events in IceCube are simulated with PROPOSAL using the same energy loss cuts as for muons.
An important difference compared to muons is, that for the propagation before the detector also the decay products which can consist of long-ranged muons, need to get stored.
Regarding searches for physics beyond the Standard Model in IceCube, PROPOSAL was used to propagate stable massive particles, like staus, with an implemented behavior similar to muons but with a mass of \SI{500}{MeV}.

Besides the neutrino telescopes detecting Cherenkov light, also the experiments of radio neutrino astronomy are using PROPOSAL for the muon and tau propagation \cite{Glaser20nuRadioMC}.
Since these detectors are searching for the highest energies and are sensitive to electromagnetic cascades above a PeV, an energy loss cut of \SI{100}{TeV} is used.
A peculiarity for the simulations of these experiments are the extremely high energies and their sensitivity calculations ranging until \SI{e30}{eV}, where also the tau leptons propagate significant ranges.
In simulation studies \cite{GarciaFernandez2020RNOG} the effect of large stochastic energy losses of muons miming neutrino events was analyzed using PROPOSAL.
Hereby, it doesn't have to be a single stochastic loss, but also the sum of several smaller energy losses inside a track segment of $\mathcal{O}(\SI{10}{m})$ have significant contributions.

%

\subsection{Air Shower Simulation} \label{sec:corsika}

In recent years, the air shower simulation framework CORSIKA \cite{CORSIKA, Engel19} with its monolithic structure written in Fortran has been restructured and rewritten into a modular structure, written in C\texttt{++}.
For this new CORSIKA 8 version, also the electromagnetic shower calculation needed a restructuring.
In the old version of CORSIKA 7, a modified version of EGS4 \cite{Bielajew94EGS4, Nelson94EGS4, EGS5} was used with additional corrections for the LPM effect.
Since EGS4 is also based on a monolithic structure, written in Fortran, PROPOSAL is used as a physics module providing the electromagnetic interaction processes.
Although PROPOSAL was more designed as a muon and tau propagator, similar cross-sections and propagation algorithms are used for the propagation of electrons, positrons, and high energy photons.
These enhancements of PROPOSAL were presented in \cite{Alameddine20PROPOSAL}.

Also, the energy region is different for some application with lower energy cuts of around \SI{1}{MeV}.
In particular, these low energetic particles are important, since the charge excess of electrons compared to positrons for the Askaryan effect is mainly driven by low energetic electrons produced in Compton scattering or ionization processes.
Furthermore, CORSIKA uses PROPOSAL in a new way.
Instead of propagating the particles for CORSIKA, PROPOSAL is proposing an interaction step along with further modules like the decay module.
Then CORSIKA determines the next step and propagates the particle itself.
Therefore the individual modules from the \texttt{propagation\_utilities} are used.

Many different modules providing physical input are required in air shower simulations, which are mainly calculating their processes independent of each other.
On the one hand, this is a necessary to keep a modular design.
On the other hand, multiple processes can affect each other, which requires approximations to be made.
This can exemplarily be described regarding the multiple scattering.
Multiple scattering is one of the main processes in electromagnetic showers responsible for the deviation of the particles from the shower axis, affecting lateral profiles significantly.
In principle, the scattering consists of a positional deviation and a change in direction.
However, due to the additional deflection induced by the magnetic field, these two deviation processes interfere with each other.
Therefore, the multiple scattering currently only changes the direction, neglecting the positional shift.

Another approximation regarding the electromagnetic propagation with PROPOSAL is the continuous energy loss, which is not calculated using the energy integral (c.f. \secref{sec:prop_util}).
Instead, the interaction point is sampled with the mean-free path length, thereby assuming no continuous energy losses.
A correction for the continuous losses is calculated in a second, independent step.

Although not all effects and corrections for the electromagnetic shower propagation are implemented yet in PROPOSAL compared to EGS4, shower distributions like the longitudinal or lateral profile were in good agreement with the old version CORSIKA 7 or other air shower frameworks, like AIRES and ZHS as presented in \cite{Alameddine21ICRC}.
However, this is an ongoing field of research where PROPOSAL will play a key role in electromagnetic shower simulations for air shower experiments like Pierre Auger, HAWC, and CTA.
Thereby, PROPOSAL can for example also provide the electromagnetic processes for hadronic particles and calculate the ionization losses.

%

\subsection{Underground Experiments and Further Application Areas}

Next to the application in new physical areas, the main usage of PROPOSAL is still the propagation of muons through large volumes.
Thereby the underground experiments with small detection volumes also start using PROPOSAL for calculating the incoming muon flux.
Since these detectors are sensitive to low energy processes, not implemented in PROPOSAL, GEANT4 is used for the propagation inside the detector.
This was exemplarily done in a simulation study for the DUNE experiment \cite{Schneider21DUNE}.

In an ongoing analysis for the PICO detector calculating the atmospheric muon flux deep underground at the detector, MCEq is used to estimate the main parts of the incoming muon distribution.
However, since the analysis with PICO is also sensitive to the rare muons in the tail of the energy distribution, a combination of MCEq and PROPOSAL as a Monte-Carlo propagator is used \cite{Fedynitch21PICO}.

Due to its simple installation and usage, PROPOSAL is not limited to being used in large simulation frameworks running on high-performance clusters.
It can also be used to produce lightweight simulation studies when only limited resources are available, like on a notebook.
For example, in \cite{GarciaFernandez2020RNOG} the effect of large stochastic energy losses of muons producing a neutrino-like radio signal is estimated.

Another advantage for BSM studies is the modular structure of the particle definition and the cross-section in PROPOSAL.
The custom particle properties can be defined and a BSM particle can be propagated with these properties, according to selected interaction processes.
For example, this has been used to calculate the sensitivity of neutrino telescopes for milicharged particles \cite{Arguelles21MiliCharged}.

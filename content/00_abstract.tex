\thispagestyle{plain}

\section*{Abstract}

Muons are the dominant particle type measured in almost every underground experiment mainly driven by the high production rate of muons in cosmic-ray induced air showers as well as the long muon range.
Due to their stochastic propagation behavior, they can remain undetected with minimal energy losses in veto regions while producing a signal-like signature with a large stochastic energy loss inside a detector.
Therefore, accurate description of theoretical models and precise treatment in simulations as well as a validation of the cross-section with measurements are required.

In this thesis, systematic uncertainties in simulations of high-energy muons were analyzed and improved, which can be divided into three parts.
The theoretical models of the cross-sections were revised and radiative corrections for the pair production interaction were calculated.
In a next step, the Monte-Carlo simulation library PROPOSAL was completely restructured in a modular design to include more accurate models and corrections.
Due to its improved usability through the modular design and its accessibility as free open-source software, PROPOSAL is now used in many applications, from large simulation frameworks, such as the CORSIKA air shower simulation, to small simulation studies.
The third part consisted of a feasibility study using PROPOSAL to measure the bremsstrahlung cross-section from the energy loss distribution, which can be measured in cubic kilometer-sized detectors.
For a detector resolution similar to that of the IceCube neutrino telescope, the bremsstrahlung normalization was estimated with an uncertainty of $\pm \SI{4}{\%}$.


\section*{Kurzfassung}
\begin{german}
Myonen sind der dominierende Teilchentyp, der in fast allen Untergrundexperimenten gemessen wird, hauptsächlich bedingt durch die hohe Produktionsrate von Myonen in durch kosmische Strahlung induzierten Luftschauern sowie die große Myonenreichweite.
Aufgrund ihres stochastischen Propagationsverhaltens können sie mit minimalen Energieverlusten unentdeckt durch Vetoregionen propagieren und innerhalb des Detektors mit einem großen stochastischen Energieverlust eine signalartige Signatur erzeugen.
Daher sind eine genaue Beschreibung der theoretischen Modelle und eine präzise Behandlung in Simulationen sowie eine Validierung des Wirkungsquerschnitts mit Messungen erforderlich.

In dieser Arbeit wurden systematische Unsicherheiten in Simulationen hochenergetischer Myonen analysiert und verbessert, was in drei Teile unterteilt werden kann.
Die theoretischen Modelle der Wirkungsquerschnitte wurden überarbeitet und Strahlungskorrekturen für die Paarproduktionswechselwirkung wurden berechnet.
In einem nächsten Schritt wurde die Monte-Carlo-Simulationsbibliothek PROPOSAL in einem modularen Design komplett umstrukturiert, um genauere Modelle und Korrekturen einbeziehen zu können.
Aufgrund der verbesserten Nutzbarkeit durch den modularen Aufbau und der Zugänglichkeit als freie Open-Source-Software wird PROPOSAL inzwischen in vielen Anwendungen eingesetzt, von großen Simulations-Frameworks, wie der Luftschauer-Simulation CORSIKA, bis hin zu kleinen Simulationsstudien.
Der dritte Teil bestand aus einer Machbarkeitsstudie unter Verwendung von PROPOSAL zur Messung des Bremsstrahlungsquerschnitts aus der Energieverlustverteilung, die in kubikkilometergroßen Detektoren gemessen werden kann.
Für eine Detektorauflösung, welcher der des IceCube-Neutrinoteleskops ähnelt, wurde die Bremsstrahlungsnormalisierung mit einer Unsicherheit von $\pm \SI{4}{\%}$ abgeschätzt.

\end{german}

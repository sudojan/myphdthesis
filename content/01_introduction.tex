\chapter{Introduction}

Muons have been first discovered in cloud chamber observations in 1936 \cite{Anderson36}.
Due to their propagated range, energy loss and deflection their signature in the detector didn't match the behaviour of an electron or a proton.
In particular it's the range muons can proapagte through large volumes of media, that makes them special and interesting for nearly all particle detectors on earth.
Muons are the only particle type from cosmic ray induced air showers that reach the surface of the earth at sea leavel.
With the high rate of cosmic rays hitting the atmosphere, their secondary muons contribute to a third of the natural radiation consumption on earth.
Even for underground experiments muons are the dominant measured event signature.

Regarding muon energies above a GeV, most muons detected at the earths surface coming from the sky and are secondary products of cosmic rays interactions.
For muons coming from down the earth, propagating horizontical or upwards, originate from neutrino interactions.
As neutrinos can propagate through the earth they can interact before the detector or inside and produce their leptonic partner, e.g. muons.

For Cosmic Ray Air shower detectors and neutrino detectors it is important to have a good understanding of the muon physics.
Next to the cross sections, also the remaining uncertainties and an understanding of their effects in analysis is important.

Since, there is no astrophysical source, sending a test beam of cosmic rays or neutrinos for all energies to calibrate the detection, Monte Carlo simulations are necessary to understand the measured data.
% Regarding the electroweak interactions of leptons and neutrinos, PROPOSAL is a simulation library for Monte-Carlo simulation providing multiple interaction and scattering models

In Chapter \ref{sec:theory} the theory is presented.
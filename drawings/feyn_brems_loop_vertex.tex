%
\begin{tikzpicture}
\begin{feynman}
    % define vertices
    % muon
    \vertex (mu_in) at (-\feynlen, 0);
    \vertex[right=3*\feynlen of mu_in] (mu_out);
    \vertex[right=0.85*\feynlen of mu_in] (mu_vertex_nucl);
    \vertex[left=1.25*\feynlen of mu_out] (mu_vertex_brems);
    % brems
    \vertex[below=0.75*\feynlen of mu_out] (brems_out);
    % nucleus
    \vertex[below=1.5*\feynlen of mu_in] (n_in);
    \vertex[below=1.5*\feynlen of mu_vertex_nucl] (n_vertex);
    \vertex[below=1.5*\feynlen of mu_out] (n_out);
    % loop
    \vertex (loop) at (mu_vertex_brems);
    \vertex (loop_left) at ($(loop) + (-180:0.45*\feynlen)$);
    \vertex (loop_right) at ($(loop) + (0:0.45*\feynlen)$);
    % draw diagram
    \diagram* {
        (mu_in) -- [fermion] (mu_vertex_nucl) -- (loop_right) -- [fermion] (mu_out),
        (mu_vertex_nucl) -- [boson] (n_vertex),
        (loop_left) -- [boson, half left] (loop_right),
        (mu_vertex_brems) -- [boson] (brems_out)
    };
    % draw extra features with tikz (not available in tikz-feynman)
    \draw[thick, double] (n_in) -- (n_vertex) -- (n_out);
    \draw[fill] (n_vertex) circle[radius=\feynvertexsize];
    \draw[fill] (mu_vertex_nucl) circle[radius=\feynvertexsize];
    \draw[fill] (mu_vertex_brems) circle[radius=\feynvertexsize];
    \draw[fill] (loop_left) circle[radius=\feynvertexsize];
    \draw[fill] (loop_right) circle[radius=\feynvertexsize];
    % add labels
    \node[left] at (mu_in) {$\mu$};
    \node[right] at (mu_out) {$\mu '$};
    \node[right] at (brems_out) {$\gamma$};
    \node[left] at (n_in) {$N$};
    \node[right] at (n_out) {$N'$};
\end{feynman}
\end{tikzpicture}
%
\hspace{1cm}
%
\begin{tikzpicture}
\begin{feynman}
    % define vertices
    % muon
    \vertex (mu_in) at (-\feynlen, 0);
    \vertex[right=3*\feynlen of mu_in] (mu_out);
    \vertex[right=1.25*\feynlen of mu_in] (mu_vertex_nucl);
    \vertex[left=0.85*\feynlen of mu_out] (mu_vertex_brems);
    % brems
    \vertex[below=0.75*\feynlen of mu_out] (brems_out);
    % nucleus
    \vertex[below=1.5*\feynlen of mu_in] (n_in);
    \vertex[below=1.5*\feynlen of mu_vertex_nucl] (n_vertex);
    \vertex[below=1.5*\feynlen of mu_out] (n_out);
    % loop
    \vertex (loop) at (mu_vertex_nucl);
    \vertex (loop_left) at ($(loop) + (-180:0.45*\feynlen)$);
    \vertex (loop_right) at ($(loop) + (0:0.45*\feynlen)$);
    % draw diagram
    \diagram* {
        (mu_in) -- [fermion] (loop_left) -- (mu_vertex_brems) -- [fermion] (mu_out),
        (mu_vertex_nucl) -- [boson] (n_vertex),
        (loop_left) -- [boson, half left] (loop_right),
        (mu_vertex_brems) -- [boson] (brems_out)
    };
    % draw extra features with tikz (not available in tikz-feynman)
    \draw[thick, double] (n_in) -- (n_vertex) -- (n_out);
    \draw[fill] (n_vertex) circle[radius=\feynvertexsize];
    \draw[fill] (mu_vertex_nucl) circle[radius=\feynvertexsize];
    \draw[fill] (mu_vertex_brems) circle[radius=\feynvertexsize];
    \draw[fill] (loop_left) circle[radius=\feynvertexsize];
    \draw[fill] (loop_right) circle[radius=\feynvertexsize];
    % add labels
    \node[left] at (mu_in) {$\mu$};
    \node[right] at (mu_out) {$\mu '$};
    \node[right] at (brems_out) {$\gamma$};
    \node[left] at (n_in) {$N$};
    \node[right] at (n_out) {$N'$};
\end{feynman}
\end{tikzpicture}
%
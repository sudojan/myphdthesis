%
\begin{tikzpicture}
\begin{feynman}
    % define vertices
    % muon
    \vertex (mu_in) at (-\feynlen-\feynsmallen, 0);
    \vertex[right=2.5*\feynlen of mu_in] (mu_out);
    % brems
    \vertex[above=\feynlen of mu_out] (brems_out);
    % blob
    \vertex[right=1.25*\feynlen of mu_in] (blob);
    \vertex (blob_left) at ($(blob) + (180:\feynsmallen)$);
    \vertex (blob_right) at ($(blob) + (0:\feynsmallen)$);
    \vertex (blob_vertex) at ($(blob) + (-90:\feynsmallen)$);
    \vertex (blob_brems) at ($(blob) + (45:\feynsmallen)$);
    % nucleus/electron target
    \vertex[below=\feynlen of mu_in] (e_in);
    \vertex[below=\feynlen of mu_out] (e_out);
    \vertex[below=\feynlen of blob] (e_vertex);
    % draw diagram
    \diagram* {
        (mu_in) -- [fermion] (blob_left),
        (blob_right) -- [fermion] (mu_out),
        (e_in) -- [fermion] (e_vertex) -- [fermion] (e_out),
        (blob_vertex) -- [boson] (e_vertex),
        (blob_brems) -- [boson] (brems_out)
    };
    % draw extra features with tikz (not available in tikz-feynman)
    \draw[pattern = north east lines] (blob) circle[radius=\feynsmallen];
    \draw[fill] (e_vertex) circle[radius=\feynvertexsize];
    % add labels
    \node[left] at (mu_in) {$\mu$};
    \node[right] at (mu_out) {$\mu '$};
    \node[left] at (e_in) {$e$};
    \node[right] at (e_out) {$e'$};
    \node[right] at (brems_out) {$\gamma$};
\end{feynman}
\end{tikzpicture}
%